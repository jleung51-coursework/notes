%
% IAT 267: Introduction to Technological Systems - A Course Overview
% Section: Actuators
%
% Author: Jeffrey Leung
%

\section{Actuators}
	\label{sec:actuators}
\begin{easylist}

	& \emph{Pulse-width modulation:} Emulation of analog output using digital-only actuators by pulsing the output on and off repeatedly
		&& The magnitude of the analog output is proportional to the amount of time the output is on relative to the amount of time the output is off

\end{easylist}
\subsection{Light and Display}
	\label{subsec:actuators:light-and-display}
\begin{easylist}

	& \hyperref[subsubsec:electricity-and-circuit-design:circuits:components]{LEDs}:
		&& LED driver chip regulates the power to a matrix (grid) of LEDs

	& \emph{Liquid Crystal Display (LCD):} Display which allows for detailed display on a screen
		&& Generally use 4 or 8 pins to input data and 3 pins for communication synchronization

\end{easylist}
\subsection{Motion}
	\label{subsec:actuators:motion}
\begin{easylist}

	& Characteristics of motors:
		&& Voltage
		&& Critical current values:
			&&& \emph{Stall current:} Amount of current used when the motor is applying maximum torque due to being held in place during operation
				&&&& Highest possible current during normal operation
			&&& \emph{Running current:} Amount of current used when the motor is activated normally
		&& Speed
		&& Torque
		&& Position (servos/steppers only)
	& Types of motors:
		&& \emph{DC motor:} Motor which has a controllable speed
		&& \emph{Servo/stepper motor:} Motor which outputs a controllable position with constant, preset speed
			&&& Controlled by the amount of time of an input pulse

\end{easylist}
\subsection{Sound}
	\label{subsec:actuators:sound}
\begin{easylist}

	& \emph{Piezospeaker:} A speaker which uses \hyperref[sec:actuators]{pulse-width modulation} and \hyperref[subsec:electricity-and-circuit-design:types-of-electricity]{piezoelectricity} to output a sound

\end{easylist}
\clearpage
