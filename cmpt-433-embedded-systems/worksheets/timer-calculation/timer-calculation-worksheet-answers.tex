%
% CMPT 433: Embedded Systems - A Course Overview
% Timer Calculation Worksheet
%
% Author: Jeffrey Leung
%

\documentclass[10pt, oneside, letterpaper]{article}

\usepackage{amsmath}
\usepackage[ampersand]{easylist}
	\ListProperties(
		Progressive*=5ex,
		Space=5pt,
		Space*=5pt,
		Style1*=\textbullet\ \ ,
		Style2*=\begin{normalfont}\begin{bfseries}\textendash\end{bfseries}\end{normalfont} \ \ ,
		Style3*=\textasteriskcentered\ \ ,
		Style4*=\begin{normalfont}\begin{bfseries}\textperiodcentered\end{bfseries}\end{normalfont}\ \ ,
		Style5*=\textbullet\ \ ,
		Style6*=\begin{normalfont}\begin{bfseries}\textendash\end{bfseries}\end{normalfont}\ \ ,
		Style7*=\textasteriskcentered\ \ ,
		Style8*=\begin{normalfont}\begin{bfseries}\textperiodcentered\end{bfseries}\end{normalfont}\ \ ,
		Hide1=1,
		Hide2=2,
		Hide3=3,
		Hide4=4,
		Hide5=5,
		Hide6=6,
		Hide7=7,
		Hide8=8 )
\usepackage{forest}
\usepackage{geometry}
	\geometry{margin=1.2in}
\usepackage{graphicx}
	\graphicspath{ {img/} }
\usepackage[colorlinks=true, linkcolor=blue]{hyperref}
\usepackage{listings}
\usepackage{lmodern} % Allows the use of symbols in font size 10; http://ctan.org/pkg/lm
\usepackage{multirow}
\usepackage{textcomp} % Allows the use of \textbullet with the font
\usepackage{verbatim}

\renewcommand{\arraystretch}{1.2}
\renewcommand{\familydefault}{\sfdefault}

\title{CMPT 433: Embedded Systems \\\medskip \Large Timer Calculation Worksheet (Answers)}
\author{}
\date{}

\begin{document}

	\maketitle
	
	\textit{Equation for calculating period of time and timer clock frequency:}
	
	\begin{equation}
		\textrm{Period} = \frac{0xFFFF\ FFFF - TLDR + 1}{\textrm{Timer Clock Frequency}} * 2^N
	\end{equation}
		
	\vspace{1cm}

	\begin{enumerate}

		\item \textbf{Given an 8-bit timer with a 32Hz clock and no divider, find the maximum timer duration.}
	
		An 8-bit timer has maximum value $0xFF$.
	
		To achieve the maximum timer duration, set $TLDR$ to $0$.
	
		The frequency is 32.
	
		The equation is:

		\begin{align}
			\textrm{Period}
			&= \frac{0xFF - TLDR + 1}{\textrm{Timer Clock Frequency}} * 2^N \\
			&= \frac{0xFF - 0 + 1}{32} * 2^0 \\
			&= \frac{2^8}{2^5} \\
			&= 2^3 \\
			&= 8s
		\end{align}
	
		Answer: The maximum timer duration is 8s.
	
		\clearpage
		
		

		\item \textbf{Given an 8-bit timer with a 32Hz clock, calculate the necessary divider and TLDR for a 200s period.}
	
		To change the period to 1s, divide the 32Hz clock by $N = 5$ to multiply the period by $2^5 = 32$.
	
		The equation is:
		\begin{align}
			\textrm{Period}
			&= \frac{0xFF - TLDR + 1}{\textrm{Timer Clock Frequency}} * 2^N \\
			&= \frac{2^8 - TLDR}{32} * 2^5 \\
			&= 2^8 - TLDR \\
			&= 256 - TLDR \\
			TLDR = 256-200 \\
			&= 56
		\end{align}
	
		Answer: $N = 0$, $TLDR = 56$
	
		\vspace{1cm}
		
		

		\item \textbf{Given a 32-bit timer with a 25MHz clock and no divider, find the maximum timer period.}
	
		A 32-bit timer has maximum value $0xFFFF\ FFFF$.

		Let $TLDR = 0$ for the maximum amount of time.
	
		The equation is:
	
		\begin{align}
			\textrm{Period}
			&= \frac{0xFFFF\ FFFF - TLDR + 1}{\textrm{Timer Clock Frequency}} * 2^N \\
			&= \frac{0xFFFF\ FFFF - 0 + 1}{25MHz} * 2^0 \\
			&= \frac{2^{32}}{25,000,000} \\
			&\approx 170s \\
			&\approx 2.8 min
		\end{align}
	
		Answer: The maximum timer duration is about 2.8 minutes.
		
		\vspace{1cm}
		\clearpage
		
		

		\item \textbf{Given a 32-bit timer with a 25MHz clock, calculate the necessary divider and TLDR for a 2s period.}
	
		Since the maximum timer period is greater than 2s (see previous question), no divider is necessary. Let $N = 0$.
	
		The equation is:
		\begin{align}
			\textrm{Period } = 2s
			&= \frac{0xFFFF\ FFFF - TLDR + 1}{\textrm{Timer Clock Frequency}} * 2^N \\
			&= \frac{2^{32} - TLDR}{25,000,000} * 2^0 \\
			50,000,000
			&= 2^{32} - TLDR \\
			TLDR
			&= 2^{32} - 50,000,000 \\
			&= 0xFD05\ 0F80 \\
		\end{align}

		Answer: $N = 0$, $TLDR = 0xFD05\ 0F80$
		
		\vspace{1cm}
		
		
		
		\item \textbf{Given the same parameters and TLDR from the previous question but with $N = 2$, how long will the period be?}
	
		If $N = 2$, the period is multiplied by $2^N = 2^2 = 4$.
		
		Period = $2s \times 4 = 8s$
	
		Answer: The period will be $8s$.

	\end{enumerate}

\end{document}
