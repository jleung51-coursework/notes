%
% CMPT 433: Embedded Systems - A Course Overview
% Bitwise Operations Worksheet
%
% Author: Jeffrey Leung
%

\documentclass[11pt, oneside, letterpaper]{article}

\usepackage{amsmath}
\usepackage[ampersand]{easylist}
	\ListProperties(
		Progressive*=5ex,
		Space=5pt,
		Space*=5pt,
		Style1*=\textbullet\ \ ,
		Style2*=\begin{normalfont}\begin{bfseries}\textendash\end{bfseries}\end{normalfont} \ \ ,
		Style3*=\textasteriskcentered\ \ ,
		Style4*=\begin{normalfont}\begin{bfseries}\textperiodcentered\end{bfseries}\end{normalfont}\ \ ,
		Style5*=\textbullet\ \ ,
		Style6*=\begin{normalfont}\begin{bfseries}\textendash\end{bfseries}\end{normalfont}\ \ ,
		Style7*=\textasteriskcentered\ \ ,
		Style8*=\begin{normalfont}\begin{bfseries}\textperiodcentered\end{bfseries}\end{normalfont}\ \ ,
		Hide1=1,
		Hide2=2,
		Hide3=3,
		Hide4=4,
		Hide5=5,
		Hide6=6,
		Hide7=7,
		Hide8=8 )
\usepackage[T1]{fontenc}
\usepackage{forest}
\usepackage{geometry}
	\geometry{margin=1.2in}
\usepackage{graphicx}
	\graphicspath{ {img/} }
\usepackage[colorlinks=true, linkcolor=blue]{hyperref}
\usepackage{listings}
\usepackage{lmodern} % Allows the use of symbols in font size 10; http://ctan.org/pkg/lm
\usepackage{multirow}
\usepackage{textcomp} % Allows the use of \textbullet with the font
\usepackage{verbatim}

\renewcommand{\arraystretch}{1.2}
\renewcommand{\familydefault}{\sfdefault}

\title{CMPT 433: Embedded Systems \\\medskip \Large Bitwise Operations Worksheet (Answers)}
\author{}
\date{}

\begin{document}

	\maketitle
	
	\textbf{Given the following predefined values where LEDs are active high and buttons are active low:}

\begin{lstlisting}[language=C, columns=fixed]
VALUE

LED0_BIT
LED1_BIT
LED2_BIT
LED_MASK = (1 << LED0_BIT) | (1 << LED3_BIT) | (1 << LED2_BIT) 

BTN0_BIT
BTN1_BIT
BTN_MASK = (1 << BTN0_BIT) | (1 << BTN1_BIT)

SPD_BIT_BEGIN
SPD_MASK
\end{lstlisting}

\vspace{1cm}

\textbf{Complete the following calculations.}

\begin{lstlisting}[language=C, columns=fixed]
_Bool isLed0On = (VALUE & (1 << LED0_BIT)) != 0;

_Bool isAnyLEDOn = ((VALUE & LED_MASK) != 0);

_Bool areAllLEDsOn = (VALUE & LED_MASK) == LED_MASK;

// If (VALUE & BTN_MASK) == BTN_MASK,
// then because buttons are active low, no buttons are pressed.
_Bool isAnyButtonPressed = (VALUE & BTN_MASK) != BTN_MASK;

_Bool areAllButtonsPressed = (VALUE & BTN_MASK) == 0;
\end{lstlisting}

\clearpage

\begin{lstlisting}[language=C, columns=fixed]
void turnOnLed0() {
	VALUE |= (1 << LED0_BIT);
}

void turnOnAllLeds() {
	VALUE |= LED_MASK;
}

void turnOffLed() {
	// ~(1 << LED_BIT) sets the LED0 bit to 0, and all
	// other bits to 1.
	// ANDing it changes the LED0 bit to 0, and does not
	// change other bits.
	VALUE &= ~(1 << LED_BIT);
}

void turnOffLeds1And2() {
	VALUE &= ~(1 << LED1_BIT | 1 << LED2_BIT);
}

void turnOffAllLeds() {
	VALUE &= ~LED_MASK;
}

void turnOffAllLedsExcept2() {
	// Remove the LED2 bit from the inverted LED mask
	VALUE &= (~LED_MASK | (1 << LED2_BIT));
}

void toggleLed0() {
	VALUE ^= (1 << LED0_BIT);
}

void toggleAllLeds() {
	VALUE ^= LED_MASK;
}

// Assume ints are in the correct format and do not need 
// to be converted to/from binary.
int getSpeed() {
	// Mask the correct bits
	// Bitshift the value so that the speed is at the least
	// significant (rightmost) bit
	return (VALUE & SPD_MASK) >> SPD_BIT;
}

int setSpeed(int speed) {
	// Shift speed to the correct location
	// Remove excess bits outside the speed bits
	int newSpeedBits = (speed << SPD_BIT) & SPD_MASK;
	// Create the value with cleared speed bits
	// Add the new speed bit
	VALUE = (VALUE & ~SPD_MASK) | newSpeedBits;
}
\end{lstlisting}

\end{document}
