%
% CMPT 433: Embedded Systems - A Course Overview
% Section: Bitwise
%
% Author: Jeffrey Leung
%

\section{Bitwise}
	\label{sec:bitwise}
\begin{easylist}

& Bitwise operators:
	&& OR (|): Set selected bits
	&& AND (\&): Clear unselected bits; only show selected
	&& NOT (\~): Invert all bits
	&& XOR (\textasciicircum): Invert selected bits
	&& Bitshifting (\lstinline[columns=fixed]{<<} for left, \lstinline[columns=fixed]{>>} for right): Moving all bits one way or another and removing the displaced digit
	&& Mask: Selects a field in a bit-flag

& Bitshifting 1 by the number of a certain position creates a mask for that position, which can be ORed to toggle it (i.e. if \lstinline[columns=fixed]{LED0_BIT == 4}, then \lstinline[columns=fixed]{1 << LED0_BIT} creates a mask at the 4th bit from the right)

& Test to check whether data fields are listed in the correct order in the data value

& Applying a mask with \& and checking whether the remaining value is equal to 0 returns whether the bit is on or off

& Given the following predefined values where LEDs are active high and buttons are active low:

\begin{lstlisting}[language=C, columns=fixed]
VALUE

LED0_BIT
LED1_BIT
LED2_BIT
LED_MASK = (1 << LED0_BIT) | (1 << LED3_BIT) | (1 << LED2_BIT) 

BTN0_BIT
BTN1_BIT
BTN_MASK = (1 << BTN0_BIT) | (1 << BTN1_BIT)

SPD_BIT_BEGIN
SPD_MASK
\end{lstlisting}

\vspace{1cm}

The calculations are as follows:

\begin{lstlisting}[language=C, columns=fixed]
_Bool isLed0On = (VALUE & (1 << LED0_BIT)) != 0;

_Bool isAnyLEDOn = ((VALUE & LED_MASK) != 0);

_Bool areAllLEDsOn = (VALUE & LED_MASK) == LED_MASK;

// If (VALUE & BTN_MASK) == BTN_MASK,
// then because buttons are active low, no buttons are pressed.
_Bool isAnyButtonPressed = (VALUE & BTN_MASK) != BTN_MASK;

_Bool areAllButtonsPressed = (VALUE & BTN_MASK) == 0;
\end{lstlisting}

\clearpage

\begin{lstlisting}[language=C, columns=fixed]
void turnOnLed0() {
	VALUE |= (1 << LED0_BIT);
}

void turnOnAllLeds() {
	VALUE |= LED_MASK;
}

void turnOffLed() {
	// ~(1 << LED_BIT) sets the LED0 bit to 0, and all
	// other bits to 1.
	// ANDing it changes the LED0 bit to 0, and does not
	// change other bits.
	VALUE &= ~(1 << LED_BIT);
}

void turnOffLeds1And2() {
	VALUE &= ~(1 << LED1_BIT | 1 << LED2_BIT);
}

void turnOffAllLeds() {
	VALUE &= ~LED_MASK;
}

void turnOffAllLedsExcept2() {
	// Remove the LED2 bit from the inverted LED mask
	VALUE &= (~LED_MASK | (1 << LED2_BIT));
}

void toggleLed0() {
	VALUE ^= (1 << LED0_BIT);
}

void toggleAllLeds() {
	VALUE ^= LED_MASK;
}

// Assume ints are in the correct format and do not need 
// to be converted to/from binary.
int getSpeed() {
	// Mask the correct bits
	// Bitshift the value so that the speed is at the least
	// significant (rightmost) bit
	return (VALUE & SPD_MASK) >> SPD_BIT;
}

int setSpeed(int speed) {
	// Shift speed to the correct location
	// Remove excess bits outside the speed bits
	int newSpeedBits = (speed << SPD_BIT) & SPD_MASK;
	// Create the value with cleared speed bits
	// Add the new speed bit
	VALUE = (VALUE & ~SPD_MASK) | newSpeedBits;
}
\end{lstlisting}

\end{easylist}
\clearpage
