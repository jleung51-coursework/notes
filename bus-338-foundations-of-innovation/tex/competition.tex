%
% BUS 338: Foundations of Innovation - A Course Overview
% Section: Competition
%
% Author: Jeffrey Leung
%

\section{Competition}
	\label{sec:competition}
\begin{easylist}

& \textbf{Competitor:} Company which has a product that, in the hands of a customer, decreases the value of your product
& \textbf{Complementor/partner:} Company which has a product that, in the hands of a customer in addition to your product, increases the value of your produc

& Alternative: Another company with the same market target
	&& Provides reference information on:
		&&& Price
		&&& Costs (sunk, running, and maintenance)
		&&& Promotion and distribution channels (current and potential)
		&&& How customers view the product
	&& Vital for pragmatists to compare against
	&& Important to position against
	&& \textbf{Market alternative:} Company currently servicing your prospective customers but which does not solve the problems you are strategically addressing
	&& \textbf{Product alternative:} Company is addressing the same problems as you and is vying for market leadership
	&& \textbf{Status quo:} Existing state without change

& Being the first to market is useful when:
	&& The product is difficult to duplicate
	&& The product satisfies customer needs well
	&& A significant proportion of the market can be quickly captured
	&& Obstacles for others can be created (e.g. regulations)

& Not being the first to market is useful when:
	&& A need is uncovered by an earlier company but not addressed
	&& Customer needs are unclear
	&& No dominant design has been accepted
	&& Switching costs are low

& Competition can be in the following factors:
	&& Brand
	&& Product attributes
	&& Quality
	&& Service
	&& Unmet needs of existing customers

& \textbf{Competitive matrix:} Analysis of customer interests addressed by your product or by competitors
	&& Categories should be accurate, actionable, and relevant
	&& Example categories:
		&&& Main customers
		&&& Key partners
		&&& Market share
		&&& Key products
		&&& Pricing
		&&& Key product features
		&&& Marketing channels
		&&& Sales channels
	&& See figure \ref{fig:competitive-matrix}

\end{easylist}

\begin{figure}[!htb]
	\centering
	\caption{Example of a Competitive Matrix}
	\label{fig:competitive-matrix}
	\begin{tabular}{ l | l l l }
		& Us & Competitor 1 & Competitor 2 \\
		\hline
		Main customers \\
		Key partners \\
		Key product features \\
		Pricing \\
		Sales channels
	\end{tabular}
\end{figure}

\clearpage
