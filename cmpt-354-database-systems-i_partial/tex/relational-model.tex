%
% CMPT 354: Database Systems I - A Course Overview
% Section: Relational Model
%
% Author: Jeffrey Leung
%

\section{Relational Model}
	\label{sec:relational-model}
\begin{easylist}

	& \emph{Relational database:} Collection of tables (mathematical concept of a relation
		&& Tables have unique names; rows represent an entity or relationship
		
	& \emph{Tuple:} Record/row of a relational database
		
	& \emph{Relation:} Set of unique tuples
		&& \emph{Relation instance:} Actual table with a particular set of rows
			&&& \emph{Cardinality (relation instance):} Number of tuples in a given relation instance
		&& \emph{Relation schema:} Column headings of a table
			&&& Consists of the names of the relation, the names of the columns, and the domain of each field
			&&& \emph{Domain:} Set of possible values
				&&&& \emph{Relation instance:} Subset of the Cartesian product of the domains; set of distinct, valid tuples/records
					&&&&& \emph{Cartesian product:} All elements in a set paired with all elements in another set
					&&&&& I.e. One possible row
			&&& E.g. Customer relation: \\
			Customer = \{ sin, firstName, lastName, age, income \} \\
			Domain: integer(9), char(20), char(20), integer, realNumber
		&& Degree (arity): Number of fields of a relation

\end{easylist}
\subsection{Relational Models vs. Databases}
	\label{subsec:relational-model:relational-models-vs-databases}
\begin{easylist}

	& For differences in terminology, see table~\ref{tab:equivalent-terms-for-relational-models-and-databases}
	
	\Deactivate
	\begin{table}[!htb]
		\centering
		\caption{Equivalent terms for Relational Models and Databases}
		\label{tab:equivalent-terms-for-relational-models-and-databases}
		\begin{tabular}{ l l }
			Relational models: & Databases: \\
			\hline
			Relation schema & Table schema \\
			Relation instance / relation & Table \\
			Field & Column/attribute \\
			Tuple & Record/row
		\end{tabular}
	\end{table}
	\Activate

\end{easylist}
\clearpage