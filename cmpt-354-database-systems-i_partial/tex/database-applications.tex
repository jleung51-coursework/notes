%
% CMPT 354: Database Systems I - A Course Overview
% Section: Database Applications
%
% Author: Jeffrey Leung
%

\section{Database Applications}
	\label{sec:database-applications}
\begin{easylist}

	& An application is built on top of the database to store, retrieve, and manipulate data as necessary
		&& Created in a general purpose programming language
		&& DBMS is a back-end which stores the data on a server
		&& May include:
			&&& Interface and presentation to provide access to data
			&&& Business logic to ensure data consistency
				&&&& Constraints which cannot be easily implemented in the DBMS
			&&& Processing and computation of data
		
	& Application architecture:
		&& \emph{Single-tier architecture:} 
			&&& DBMS, application logic, business rules, and user interface in the same tier
			&&& Run on a mainframe; accessed through dumb terminals
				&&&& \emph{Dumb terminal:} Terminal which lacks the computational power to support a GUI
		&& \emph{Two-tier architecture:} Used for client-server architecture
			&&& Consist of client and server computers
				&&&& \emph{Thin client:} Client which implements the UI
					&&&&& Commonly used
				&&&& \emph{Thick client:} Client which implements the UI and some of the business logic
					&&&&& Uncommonly used
					&&&&& More difficult to update and maintain the business logic
					&&&&& Do not scale well
					&&&&& %TODO
				&&&& Server implements the business logic and data management; clients are not responsible for data processing
		&& \emph{Three/Multi-tier architecture:} Used for web systems
			&&& Scalable
			&&& Presentation layer handles the user interaction with the middle? tier
				&&& Differnent interfaces... %TODO
			
				&&& \emph{Style sheet:} Method to format a document in different ways depending on the context (e.g. mobile devices)
					&&&& Display format can be standardized with the same conventions
			&&& \emph{Business logic layer:} Runs the business logic of the application
				&&&& Controls:
					&&&&& %TODO
			&&& \emph{Data management layer:} Contains the database(s)
				&&&& Data is exchanged between the middle tier and the database servers if  multiple data sources are used
					&&&&& XML is often used as a data exchange format between database servers and the middle tier
			&&& State of a user's interaction must be maintained across different web pages
				&&&& Maintained at the middle tier in... %TODO
			&&& Advantages:
				&&&& \emph{Heterogeneous system:} 
					&&&&& Applications can use different platforms and software components at different tiers
					&&&&& Ease of modifying code at any specific tier
				&&&& %TODO
				
	& Database must be connected to the host language (the programming language that the application is written in)
		&& Database connections allows communiation between a database server and client software
		&& %TODO
		&& Accessing database data from a host language application:
			&&& \emph{Embedded SQL:} SQL statements identified at compile time
				&&&& Does not allow the creation of new SQL queries
				&&&& SQL statements are marked so the preprocessor can process them
				&&&& Variables used to pass arguments into a SQL command are declared in SQL
					&&&&& Data types may not match
					&&&&& E.g. C:
					\begin{lstlisting}[language=C]
EXEC SQL BEGIN DECLARE SECTION;
long sin;
char[20] fname;
char[20] lname;
int age;
EXEC SQL END DECLARE SECTION;
					\end{lstlisting}
					
					\begin{lstlisting}[language=C]
EXEC SQL
INSERT INTO Customer VALUES
	(:sin, :fname, :lname, :age);
					\end{lstlisting}
					
				&&&& SQLSTATE: Array of 5 characters which connects the host language to... %TODO
				&&&& \emph{Cursor:} Position in a relation which allows retrieval of a set of records
					
			&&& \emph{Dynamic SQL:} SQL queries constructed at runtime
				&&&& E.g. Open DataBase Connectivity, Java DataBase Connectivity - Protocols allownig a single executable to be run on different databases without... %TODO
				&&&& New queries can be created
				&&&& Query is created as a string, then parsed and executed
			&&& Mapping a database table to a host language construct
				&&&& E.g. CRecordSet
			&&& %TODO

\end{easylist}
\section{Object Relational Mapping}
	\label{subsec:databse-applications:object-relational-mapping}
\begin{easylist}

	& \emph{Object relational mapping:} Process of converting data between an object/class and attributes in a database
	
	& \emph{Impedance mismatch:} System where inputs and outputs do not match
	
	& Tables are mapped to classes; columns are mapped to member variables
	& Inheritance options:
		&& Table for each class
		&& Table for each concrete class
		&& Table for each class family
	& Mapping a subclass to a table:
		&& If each class in an inheritance
	
	& Complex data types are composed of composite/multi-valued attributes

\end{easylist}
\subsection{Injection}
	\label{subsec:database-applications:injection}
\begin{easylist}

	& \emph{SQL injection:} Code injection attack which places unwanted code into an SQL query
	
	& E.g. Given the following query in response to entered data:
	\begin{lstlisting}
SELECT attribute
FROM table
WHERE attribute = '$email'  -- Variable from the form
	\end{lstlisting}
		
		&& If the input is:
		\begin{lstlisting}
X'; DROP TABLE tablename --
		\end{lstlisting}
		then the entire table would be deleted.
		
		&& If the input is:
		\begin{lstlisting}
anything' OR 'X' = 'X
		\end{lstlisting}
		then the application may display an email in the table (often the first one).
		
		&& If the input is:
		\begin{lstlisting}
X'; UPDATE members SET email = 'myemail@gmail.com' WHERE email = 'someonesemail@gmail.com' --
		\end{lstlisting}
		then the email for an account will be changed to another email and access to the account will be compromised.
		
	& Preventing injection attacks:
		&& \emph{Sanitized input:} Data which has no invalid characters
			&&& Difficult due to exceptions such as apostrophes and hyphens in names
		&& \emph{Bound input:} Data which is given to the code through placeholders
			&&& Can be done by the \emph{Prepare} statement
		&& Limit permissions and segregate users
		&& Use stored procedures

\end{easylist}
\clearpage