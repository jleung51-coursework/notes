%
% CMPT 354: Database Systems I - A Course Overview
% Section: Entity-Relationship Model
%
% Author: Jeffrey Leung
%

\section{Entity-Relationship Model}
	\label{sec:entity-relationship-model}
\begin{easylist}

	& \emph{Entity-relationship model:} Conceptual visualization of a database' structure through the use of entities and their relationships
		&& Used to create a high-level description of the data
	& \emph{Entity-relationship diagram:} Visual representation of an entity-relationship model
		&& Many alternative ways to be designed

\end{easylist}
\subsection{Components}
	\label{subsec:entity-relationship-model:components}
\begin{easylist}	
	
	& \emph{Entity:} Distinguishable, distinct object
		&& Concrete or abstract
		&& Described by a set of attributes
		&& \emph{Attribute:} Information about an entity
			&&& \emph{Composite:} Attribute composed of multiple different sub-attributes
				&&&& E.g. An address is composed of a number, street name, and city
			&&& \emph{Multi-valued:} Attribute with multiple sub-attributes of the same type
				&&&& E.g. Phone numbers for home, work, cell
				&&&& Should be replaced with an entity set, with relationships used to connect with its entity
			&&& \emph{Derived:} Attribute which contains information from related attribute(s)
				&&&& E.g. Total number of employees
				&&&& Do not need to be stored in the database; can be immediately derived when necessary
			&&& \emph{Foreign key:} Attribute which references a (primary) attribute from another table
				&&&& Can be explicitly declared as non-null
			&&& May have a null value
			&&& Has a domain (set of possible attribute values)
		&& \emph{Entity set:} Collection of entities under some categorization
			&&& Not necessarily disjoint (multiple entity sets may contain the same entity)
			&&& Cannot contain multiple of the same entity


	& \emph{Relationship:} Association between entities
		&& E.g. Student \emph{enrolled} in a course
		&& Constraint: Rule or policy which restricts a relationship
		&& \emph{Relationship set:} Set of related relationships
			&&& Often represented with a verb/action
			&&& Attributes:
				&&&& Primary key depends on the key constraints of the relationship and the primary keys of its connected entities (not shown in an entity-relationship diagram)
				&&&& \emph{Descriptive attribute:} Detail of a relationship set
					&&&&& Not usable to identify the relationship set with
					&&&&& Cannot be part of the relationship set's primary key
			&&& Used for properties which cannot be associated with any of the entity sets
			&&& An $n$-ary relationship set $R$ is an association between $n$ entity sets $E_{1} \cdots E_{n}$
				&&&& $R$ is a subset of $\{ (e_{1}, \ldots, e_{n}) | e_{1} \in E_{1}, \ldots, e_{n} \in E_{n}\}$
			&&& Role of an entity is the function it plays in the relationship
			&&& E.g. Employee (entity) works (relationship) on a project (entity). \\
			`Working on' is a relationship set, as it can be applied to to many employees working on projects. \\
			The start date of the employee working on the project is a descriptive attribute.
			&&& E.g. Manager (entity) manages (relationship) an employee. \\
			`Manages' is a relationship set, as it can be applied to many managers managing employees. \\
			The superiority of the manager over the employee is a descriptive attribute.
			
			&&& Mapping cardinalities:
				&&&& Specify how many of each entity may be related to another entity
				&&&& Types of mapping between sets:
					&&&&& \emph{One-to-one:} Each entity in A maps to at most one entity in B; each entity in B maps to at most one entity in A
						&&&&&& E.g. One employee is only assigned one office; one office can only hold one employee
					&&&&& \emph{One-to-many:} Each entity in A may map to any number of entities in B; each entity in B maps to at most one entity in A
						&&&&&& E.g. A person can own many bank accounts; a bank account can only be owned by one person
					&&&&& \emph{Many-to-one:} Each entity in A maps to at most one entity in B; each entity in B may map to any number of entities in A
						&&&&&& E.g. A player can be in only one soccer team; a soccer team can contain many players
					&&&&& \emph{Many-to-many:} Each entity in A may map to any number of entities in B; each entity in B may map to any number of entities in A
						&&&&&& E.g. A person can be enrolled in many courses; a course can contain many students
				&&&& In an entity-relationship diagram: directed lines
					&&&&& A directed line from an entity set to a relationship set indicates that the entities in the set can only be involved in one relationship in the set
			&&& \emph{Participation constraint:} Rule that any entity in some entity set must be involved in a relationship(s)
				&&&& Examples:
					&&&&& An account must be owned by at least one customer; a customer does not have to have an account
					&&&&& A child must attend a single pre-school, which must have at least one child
					&&&&& A branch of a bank must contain at least one account; each account must be at only one branch
				&&&& \emph{Total participation:} A participation constraint exists
				&&&& \emph{Partial participation:} A participation constraint does not exist
				&&&& In an entity-relationship diagram: Double line between the relationship and entity
			&&& \emph{Key constraint (entity-relationship model):} An entity can participate in only one relationship
		
		&& Does not have to be binary
			&&& Ternary relationship example: A branch purchases a part from a supplier 
	
\end{easylist}
\subsection{Additional Features}
	\label{subsec:entity-relationship-model:additional-features}
\begin{easylist}
		
	& \emph{Weak entity set:} Entity set containing entities which cannot be identified using only their own attributes
		&& Identification: Compound key of its partial key with the primary key of another entity set
		&& Total participation and key constraint in the identifying relationship
		&& \emph{Owner entity set:} Entity set containing a primary key which is combined with a partial key of a weak entity set to identify a weak entity
		&& In an entity-relationship diagram: Double-line from the weak entity set to the relationship set; double-border around the relationship set
		
	& \emph{Subclass/Superclass:} Hierarchy of entity sets which are contained within other entity sets, and vice versa
		&& Used to:
			&&& Avoid redefining common attributes
			&&& Add additional descriptive attributes
			&&& Identify the set of entities which participate in a particular relationship
		&& Subclasses are defined by its attributes and the attributes of its superclass(es)
		&& May have its own additional attributes
		&& `is a' relationship
			&&& E.g. A (subclass) is a B (superclass); A specializes/generalizes B
		&& \emph{Specialization:} Identification of subsets with additional attributes from an existing entity set
		&& \emph{Generalization:} Identification of common attributes of entity sets and creating a new entity set with the attributes
		&& \emph{Condition-defined participation:} A given entity is in a subclass if it meets a specific condition
		&& \emph{User-defined participation:} A given entity is in a subclass if it is assigned as such by the user
			&&& E.g. Employees must be manually promoted to manager
		
		&& \emph{Disjoint:} Subclass whose entities cannot appear in another subclass
			&&& E.g. Birds and mammals
			&&& Assumed if unlabeled
		&& \emph{Overlapping:} Subclass whose entities may appear in another subclass
			&&& E.g. Customer and employee
			&&& Must be labeled
		
		&& \emph{Coverage constraint:} Each superclass entity must belong to a subclass
			&&& E.g. Animals must belong to a kingdom
		
	& \emph{Aggregation:} Using a relationship set as an entity set to assign it to another relationship
		&& Used when there is a relationship between an entity set and another relationship
		&& E.g. A branch can only buy the same part from the same supplier; the aggregate entity is the branch buying the part, and there is the relationship where the aggregate entity buys from the supplier
	
\end{easylist}
\subsection{Design Principles}
	\label{subsec:entity-relationship-model:design-principles}
\begin{easylist}

	& Faithfulness:
		&& Design must be accurate to the specification and the enterprise
		&& All relevant aspects must be represented
	& Simplicity:
		&& Redundancy confuses understanding, wastes storage, creates inconsistency
		&& Use attributes over entity sets or relationships
		&& Specify as many constraints as possible
	& Element choices:
		&& Entity set vs attribute
		&& Entity set vs relationship set
		&& Type of relationship:
			&&& Multiple binary relationships
			&&& n-ary relationship
			&&& Aggregation
	& Relationship sets' attributes must be descriptive (i.e. cannot be a part of the primary key)
	
\end{easylist}
\subsection{Converting Entity-Relationship Diagrams to a Database}
	\label{subsec:entity-relationship-model:converting-entity-relationship-diagrams-to-a-database}
\begin{easylist}

	& Each entity and relationship set is represented by a unique relation
	& Constraints in the database should be created to match constraints in the model
	& Primary keys are implicitly NOT NULL
	
	& Table representing an entity set:
		&& A column for each attribute
		&& The domain of each attribute should be specified
		&& Primary key should be specified
	& Relationship sets:
		&& Attributes:
			&&& Primary keys of the entity sets (declared as foreign keys)
			&&& Descriptive attributes
		&& Mapping cardinalities determine the primary key of the relationship, as well as whether the relationship set requires a separate table or just foreign keys with constraints
			&&& No cardinality constraint (many-to-many) (n-ary relationship set):
				&&&& Requires a separate table
				&&&& Primary key: Compound key; union of the attributes from the entity sets (declared as foreign keys)
				&&&& Descriptive attributes declared as non-foreign keys
			&&& Many-to-one constraint (n-ary relationship set):
				&&&& Represented with either a table or an attribute of an entity set
				&&&& Primary key: Primary key of (one of) the entity from the `many' entity set(s)
			&&& One-to-one constraint:
				&&&& Primary key: One and only one primary key of the entity sets
		&& Participation constraint:
			&&& Declare the attribute on which a foreign key exists to be NOT NULL
				&&&& Does not work if a relationship set is not represented in a separate table
			&&& May require assertions/triggers
		
	& Weak entity set:
		&& Must have a foreign key referencing its owner's primary key
		
	& Class hierarchy: (two ways)
		&& Create separate tables for the superclass and subclasses; make both a superclass and subclass tuple for a given subclass instances
			&&& Subclass tables only have the attributes which the superclasses do not have
			&&& Subclass tables have a foreign key referencing the superclass entity
			&&& Deletion should cascade from superclass records to subclass records
		&& Create tables for subclasses only
			&&& Coverage constraint must exist (all superclass entities must be also subclass entities)
			&&& Subclass tables contain all attributes of the superclass
	
	& Aggregation:
		&& Defined by the relationship set connecting the aggregate entities, as well as another relationship set connected to another entity which references the aggregate entities
		&& Relationship between the aggregate entities contains:
			&&& Primary key of the participating entity set
			&&& Primary key of the relationship defining the aggregation
			&&& Descriptive attributes
		&& If there is total participation between the aggregate entity and its relationship, and there are no descriptive attributes: \\
			Move the aggregate entities' relationship attributes into the table representing the relationship with the aggregate entity
	
\end{easylist}
\subsection{Entity-Relationship Diagrams in CMPT 354}
	\label{subsec:entity-relationship-model:entity-relationship-diagrams-in-cmpt-354}
\begin{easylist}

	& Entity: Rectangle
		&& Attribute: Oval
			&&& Linked to their entity with a line
			&&& Derived attribute: Oval with a broken outline
			&&& Attribute of a primary key is underlined
	& Relationship: Line
		&& Relationship set: Line with a description in a diamond
			&&& X-to-one relationship: Directed line pointing from the entity with restrictions to the relationship set
		&& Participation constraint: Double/thick line
		&& Subclass/superclass: Triangle with the words `is a', the superclass connected above, and the subclass(es) connected below
			&&& Disjoint: Default, unmarked
			&&& Overlapping: Labeled as `OVERLAP'
			&&& Coverage constraint: Labeled as `COVER'
	& Aggregation: Rectangle around the entity sets and relationship set; rectangle as a whole is treated as an entity set

\end{easylist}
\subsection{Example Diagram}
	\label{subsec:entity-relationship-model:example-diagram}

See figure~\ref{fig:example-entity-relationship-diagram}.

	\begin{landscape}
		\begin{figure}[!htb]
			\centering
			\caption{Example Entity-Relationship Diagram}
			\label{fig:example-entity-relationship-diagram}
			
			\begin{tikzpicture}
				[
					node distance = 1.5cm
				]
				
				\node[entity]
					(entity-1)
					{Entity};
					
					\node[attribute]
						(attribute-1)
						[above=of entity-1]
						{\key{Primary key attribute}}
						edge(entity-1);
					
					\node[attribute]
						(attribute-2)
						[left=of entity-1]
						{Attribute}
						edge(entity-1);
					
				\node[relationship]
					(r-e1-e2)
					[below=of entity-1]
					{Relationship}
					edge(entity-1);
					
				\node[entity]
					(entity-2)
					[below=of r-e1-e2]
					{Entity}
					edge(r-e1-e2);
					
				% Aggregation
				\node
					[
						rectangle,
						draw=gray,
						fit=(entity-1)(r-e1-e2)(entity-2),
						inner xsep=0.5cm
					]
					{};
					
				\node[relationship]
					(r-e1-e3)
					[right=of entity-1]
					{Relationship}
					edge[total](entity-1);
					
				\node[entity]
					(entity-3)
					[right=of r-e1-e3]
					{Superclass entity}
					edge[<-](r-e1-e3);
					
					\node[derived attribute]
						(attribute-3)
						[right=of entity-3]
						{Derived attribute}
						edge(entity-3);
					
				\node[isa]
					(isa-1)
					[below=of entity-3]
					{is a}
					edge(entity-3);
					
				\node[entity]
					(entity-4)
					[below=of isa-1]
					{Subclass entity}
					edge(isa-1);
				
			\end{tikzpicture}
		\end{figure}
	\end{landscape}

\clearpage