%
% CMPT 354: Database Systems I - A Course Overview
% Section: Introduction
%
% Author: Jeffrey Leung
%

\section{Introduction}
	\label{sec:introduction}
\subsection{Basic Terms}
	\label{subsec:introduction:basic-terms}
\begin{easylist}

	& \emph{Database:} Collection of information
	& \emph{Database (Computing science):} Collection of data
		&& Composed of:
			&&& \emph{Database Management System (DBMS):} Software which stores, manages, and retrieves information in a database
				&&&& Uses a relational model to represent data
				&&&& Advantages of using a DBMS to record data:
					&&&&& Efficient access:
						&&&&&& Index structure maps the attribute values to the address of the record
						&&&&&& Records can be retrieved without scanning all the data
					&&&&& Transactions (a set of actions) are processed so the integrity of the database is maintained, regardless of the number of concurrent users or a system crash
				&&&& Diskspace manager (storage): Interacts with the OS file system and data
					&&&&& Allows the DBMS to consider the data as a collection of pages
				&&&& Buffer manager: Brings pages from the disk into main manager
					&&&&& Manages policy when the main memory is full
		
	\medskip
	& Databases control all interaction between the data and the application(s)
	& Advantages of a database:
		&& Allows synchronization of access/management/modification of data
		&& Logical and physical data independence
		&& Reduced application development time
		&& Efficient access
		&& Concurrent access/control
		&& Data integrity and security
		&& Crash recovery
	
\end{easylist}
\subsection{History}
	\label{subsec:introduction:history}
\begin{easylist}

	& Originally used by large organizations to store textual data
	& 1975: 300 databases containing about 50 million records
	& 1998: 11,000 databases holding 12 billion records
	& Today:
		&& Huge amounts of data processed
			&&& E.g. Walmart handles more than 1 million transactions per hour
		&& 2.7 zettabytes of data in the world
			&&& Facebook stores over 30 petabytes of user-generated data
			
\end{easylist}
\subsection{Data Abstraction}
	\label{subsec:introduction:data-abstraction}
\begin{easylist}

	& \emph{Data model:} Formal language for describing data
		&& \emph{Schema:} Description of a particular collection of data using a particular data model
		&& \emph{Relational data model:} %TODO definition
			&&& See section~\ref{sec:relational-model}

	& 3 levels of abstraction (from lowest-level to highest-level):
		&& \emph{Physical schema:} Description of how the data is stored and indexed
			&&& Connects the database with the contextual/logical schema
			&&& E.g. Relational databases, document-store databases
		&& \emph{Contextual/logical schema:} Description of what type of data is stored in terms of the data model
			&&& Connects the physical schema and the external/view schema
			&&& E.g. Attributes of a table
		&& \emph{External/view schema:} Description of how the data is displayed and accessed
			&&& Displays select information from the contextual/logical schema for the client
			&&& \emph{View:} A selective display of data
				&&&& Different users may see different views of the same data
		\smallskip
		&& Allows:
			&&& Modification of one level without affecting the other levels
			&&& Data independence:
				&&&& \emph{Physical data independence:} Ability to modify the physical schema without rewriting the application programs
					&&&&& Useful for performance improvements (e.g. moving a file, or adding or removing an index)
				&&&& \emph{Logical data independence:} Ability to modify the logical schema without rewriting the application programs or modifying the views
					&&&&& E.g. Adding an attribute to a relation
	
	& \emph{Transaction:} Single execution of a user program in a DBMS
		&& E.g. Transferring money between accounts, looking up balance
		&& May result in multiple modifications/actions
			&&& May be interleaved with actions from another transaction
		&& Properties:
			&&& (Being) \emph{Atomic:} Either none or all of a transaction's actions are executed
				&&&& Transactions are considered in their entirety
				&&&& Avoids leaving the database in an inconsistent state due to interrupted transactions
			&&& (Having) \emph{Consistency:} Transactions should preserve the rules of the database
				&&&& E.g. Entering a patient record with a duplicate number would leave the database inconsistent, so the database should reject the transaction
				&&&& E.g. Customers require an account; accounts require a customer - database cannot maintain the circular logic, so an application program must maintain this
				&&&& E.g. Employers cannot be customers
				&&&& Databases remain consistent if transactions are:
					&&&&& Consistent
					&&&&& Processed equivalently to some serial order
					&&&&& Atomic
				&&&& Must satisfy all constraints of the database (e.g. example key constraints, foreign key constraints)
				&&&& Generally cannot be detected by the DBMS; inconsistencies may relate to domain-specific information
				&&&& Users and programs which access the database are responsible for transaction consistency
					&&&&& Casual users should only use programs which preserve consistency in the database
			&&& (Makes sense in) \emph{Isolation:} Transactions should stand on their own and be protected from the effects of other concurrent transactions
				&&&& End result should be equivalent to processing the transactions in serial order
			&&& (Being) \emph{Durable:} Effects of a transaction should persist afterwards even in the event of a crash
				&&&& Transactions are processed in main memory and later written to the disk
				&&&& Log is maintained which records information about database writes
					&&&&& Used to restore transactions not written to disk if a system crashes
				&&&& Backups must be maintained
			
\end{easylist}
\subsection{Languages}
	\label{subsec:introduction:languages}
\begin{easylist}

	& Database languages are divided into two parts:
		&& \emph{Data definition language (DDL):} Programming language used to create a database and its integrity constraints, and modify external/conceptual schema
			&&& Used to create:
				&&&& Databases
				&&&& Specification of integrity constraints
					&&&&& E.g. Domain constraints, assertions, authorizations
		&& \emph{Data manipulation language (DML):} Programming language used to access/change data in a database
			&&& Modifies database schema such as:
				&&&& Adding new tables
				&&&& Deleting tables
				&&&& Adding attributes
			&&& \emph{Procedural:} Data manipulation language in which users specify what data is required and its method of retrieval
			&&& \emph{Declarative (nonprocedural):} Data manipulation language in which users specify what data is required, but not its method of retrieval
			
\end{easylist}
\subsection{Types of Databases}
	\label{subsec:introduction:types-of-databases}
\begin{easylist}

	& Relational database management systems:
		&& Expresses relationships using foreign keys
		&& Advantages:
			&&& Access to persistent data
			&&& Concurrent access
			&&& %TODO
		&& Disadvantages:
			&&& Mismatch with object-oriented programming (partially mitigated by object-relational mapping frameworks)
			&&& Not designed to run on clusters (see %TODO ~\ref{})
			
	& Alternatives to RDBMSes which do not use SQL: Amazon Dynamo, Google BigTable
			
	& Separate application databases:
		&& Integration with web services
		&& See section~\ref{sec:nosql}
		&& Advantages:
			&&& Flexible
				&&&& Data can be added as required
				&&&& Un-needed columns do not have to be deleted
			&&& Ease of manipulating non-uniform data
		&& Disadvantages
			&&& Most programs rely on schemata
				&&&& Implicit schema must be deduced
				&&&& Essentially, the schema is moved from the database to the application
			
\end{easylist}
\clearpage