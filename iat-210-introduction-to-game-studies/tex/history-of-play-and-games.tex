%
% IAT 210: Introduction to Game Studies - A Course Overview
% History of Play and Games
%
% Author: Jeffrey Leung
%

\section{History of Play and Games}
	\label{sec:history-of-play-and-games}
\subsection{General Background}
	\label{subsec:history-of-play-and-games:general-background}
\begin{easylist}

	& \emph{Game:} Series of interesting, meaningful choices in pursuit of a clear and compelling goal
		&& Definitions from various experts:
			&&& A game is the voluntary effort to overcome unnecessary obstacles. \emph{-- Bernard Suits}
			&&& A game is an activity among two or more decision makers seekking to achieve their objectives in a limiting context. \emph{-- Clark Abt}
			&&& A game is an art in which players make decisions in order to manage resources in pursuit of a goal. \emph{-- Greg Costikyan}
			&&& A game is an exercise of voluntary control systems in which there is a contest of power, confined by rules, to produce a disequilibrial outcome. \emph{-- Brian Sutton-Smith}
			&&& A formal game has a two-fold structure based on ends (a winning condition) and means (rules by which you win). \emph{--David Parlett}
	
	& \emph{Demystification:} Games' origins are often rooted in traditions
		&& E.g. Masks were sacred objects, checkers was for divination
		
	& Play, fun, and games:
		&& \emph{Homo Ludens} by Johan Huizinga discusses the importance of play in society
		&& Play is a necessary condition for civilization
			&&& E.g. Release of tensions, diplomacy
		&& Characteristics of play:
			&&& Free activity
			&&& Fun; not serious; no material interest
			&&& Bounded; separate from the real world
			&&& Immersive

	& \emph{Gary Gygax:}
		&& Avid wargamer
		&& Responsible for developing:
			&&& The D20 combat system
			&&& The ruleset for \emph{Chainmail}, the origin of \emph{Dungeons and Dragons}
			&&& \emph{Dungeons and Dragons:} Tabletop roleplaying game
				&&&& Players assume the roles and attributes of fantasy characters who embark on magical adventures

\end{easylist}
\subsection{Digital Games}
	\label{subsec:history-of-play-and-games:digital-games}
\begin{easylist}
				
	& Arcade games:
		&& 1890s: Simply streetside amusements
		&& 1940s: Electromechanical projection machines became popular after WWII
		&& 1950s: Pinball games developed
		&& 1970-80s: Arcades became prevalent in malls
		
	& \emph{Nolan Bushnell:}
		&& Creator of Chuck E Cheese's
		&& Invented the Atari game system
		&& Engineer to entrepreneur
		&& Employed Steve Wozniak and Steve Jobs, who:
			&&& Originally worked with Atari on the arcade game \emph{Breakout}
			&&& Created the first Apple computer with Atari parts
			&&& Asked Bushnell to invest \$50,000 in their startup, who declined
		&& Inspired by B.F. Skinner:
			&&& Coined operant conditioning
				&&&& Giving consistent positive/negative feedback to someone to condition them to act that way more/less
			&&& \emph{Skinner box:} Psychological tool which dispensed rewards when a test subject activated a lever
				&&&& Led to one-armed bandits (gambling machines which only require pulling a lever)
	
	& Personal computers created possibilities for game developers
	& \emph{Johann von Neumann:} Contributor to the digital computer
		&& Played games such as kriegspiel (miniature war re-enactments)
		&& Began a more formal analysis of parlour games, leading to the study of game theory itself with a wide range of applications
		&& Defined games as conflicts between players:
			&&& Players are assumed to be 'rational' (i.e. will pursue maximum utility)
			&&& \emph{Minimax theorem:} There is always a rational solution to games where players are in conflict
			&&& Solutions can result in a positive/negative/zero sum game

\end{easylist}
\clearpage