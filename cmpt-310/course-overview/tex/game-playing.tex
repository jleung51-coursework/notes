%
% CMPT 310: Artificial Intelligence - A Course Overview
% Section: Game Playing and Adversarial Search
%
% Author: Jeffrey Leung
%

\section{Game Playing and Adversarial Search}
	\label{sec:game-playing-adversarial-search}
\begin{easylist}

& Game playing: Scenario where a game with discrete states is played with two players, each seeking to maximize their position which minimizes the opponent's position
	&& E.g. Chess, checkers
	&& \textbf{Heuristic/evaluation function:} Estimate of the optimality of a game state
	&& \textbf{Terminal state:} Game state which has no further states
	&& Example diagram of game state evaluations in a tree: See figure~\ref{fig:game-tree}
	
\begin{figure}[!htb]
	\caption{Game Tree (Minimizer first)}
	\label{fig:game-tree}
	\centering

	\begin{forest}
		for tree={
			draw,
			circle,
			minimum size=1cm,
			line width=0,
			align=center
		},
		[0
			[2
				[2]
				[-3]
				[-14]
			]
			[14
				[14]
				[17]
				[20]
			]
			[0
				[-12]
				[-18]
				[0]
			]
		]
	\end{forest}
\end{figure}

& \textbf{Minimax algorithm:} Decision-making rule which minimizes the potential loss for a worst-case scenario, given two entities who are alternatively trying to maximize and minimize a score
	&& Performs the equivalent of a depth-first search
	&& Each alternating level of the decision tree is minimized or maximized from the values in the levels below it
	&& Complete: Yes, as long as the tree is finite

& \textbf{$\alpha / \beta$ pruning:} Keeping the maximum/minimum searched values at a given level so far to eliminate nodes which will not exceed a minimum/maximum
	&& $\alpha$: Lower bound on the potential value of a maximum node
	&& $\beta$: Upper bound on the potential value of a minimum node
	&& \href{http://inst.eecs.berkeley.edu/~cs61b/fa14/ta-materials/apps/ab_tree_practice/}{Resource for practice}
	&& See figure~\ref{fig:game-tree-ab} for an example of pruned nodes (notated by dashed edges)
	
\begin{figure}[!htb]
	\caption{Game Tree with $\alpha / \beta$ Pruning (Minimizer first)}
	\label{fig:game-tree-ab}
	\centering

	\begin{forest}
		for tree={
			draw,
			circle,
			minimum size=1cm,
			line width=0,
			align=center
		},
		[0
			[2
				[2]
				[-3]
				[-14]
			]
			[14
				[14]
				[17, edge={densely dashed}]
				[20, edge={densely dashed}]
			]
			[0
				[-12]
				[-18]
				[0]
			]
		]
	\end{forest}
\end{figure}

& Time, space, and optimal complexity of algorithms: See figure~\ref{fig:algos-complexity}

\end{easylist}
\clearpage
