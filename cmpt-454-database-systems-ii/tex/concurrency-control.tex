%
% CMPT 454: Database Systems II - A Course Overview
% Section: Concurrency Control
%
% Author: Jeffrey Leung
%

\section{Concurrency Control}
	\label{sec:concurrency-control}
\begin{easylist}

& Concurrency control:
	&& If atomicity is maintained and there is no system crash, then concurrency control guarantees consistency and isolation

\end{easylist}
\subsection{Consistency}
	\label{subsec:consistency}
\begin{easylist}

& \textbf{Serial schedule:} Consistent schedule where transactions are executed sequentially
	&& Different transaction orders may create different DB states, all of which are acceptable
	&& Example: Given transactions:
	\end{easylist}
	\begin{align*}
		T1: & \ R1(A) \ W1(A) \\
		T2: & \ R2(A) \ W2(A)
	\end{align*}
	\begin{easylist}
	The two possible serial schedules are:
	\end{easylist}
	\begin{align*}
		T1\ T2 = & \ R1(A)\ W1(A)\ R2(A)\ W2(A) \\
		T2\ T1 = & \ R2(A)\ W2(A)\ R1(A)\ W1(A)
	\end{align*}
	\begin{easylist}

& \textbf{(View) Serializable schedule:} Consistent schedule where, for every data item, all transaction read and final write actions have the same effect as a serial schedule of the same transactions

	&& Example: Given transactions:
	\end{easylist}
	\begin{align*}
		T1: & \ R1(A) \ W1(A)\ R1(B)\ W1(B) \\
		T2: & \ R2(B) \ W2(B)
	\end{align*}
	\begin{easylist}
	
	A serializable schedule is:
	\end{easylist}
	\begin{align*}
		R1(A)\ W1(A)\ R2(B)\ W2(B)\ R1(B)\ W1(B)
	\end{align*}
	\begin{easylist}
	
	A non-serializable schedule is:
	\end{easylist}
	\begin{align*}
		R1(A)\ W1(A)\ R1(B)\ R2(B)\ W1(B)\ W2(B)
	\end{align*}
	\begin{easylist}
	
	&& Explanation: For each table in each transaction, the final write operation is dependent on all previous reads in the same transaction. \\
	Therefore, if for every data item in every transaction in a given schedule, the read operations and final write operation have the same effect as in a serial schedule, then the schedules are equivalent.
	
	
	&& Manual method to determine if a schedule is serializable, given a set of transactions:
		&&& Write out all possible serial schedules
		&&& For each serial schedule possibility:
			&&&& For each table:
				&&&&& In the test schedule and the current serial schedule, isolate the read operations and final write operation on the current table
				&&&&& Compare the isolated operations
				&&&&& If any operation does not have the same effect, then move to the next serial schedule possibility
			&&&& If all tables' read and final write operations have the same effect as the current serial schedule, then the schedule is serializable
		&&& If no serial schedule possibility matches, then the schedule is not serializable

	&& \textbf{Conflict-based testing:} Method to determine if a schedule is serializable through analyzing its conflict actions
	
		&&& Actions from an aborted transaction are ignored when calculating conflicts
		
		&&& \textbf{Conflict actions:} Two subsequent read/write actions (at least one write) from different transactions (in a schedule) which operate on the same data item
			&&&& Every conflict action \textbf{induces} the order of transactions to match the order in the two operations
		
			&&&& Types of conflict actions:
				&&&&& If the operations are write then read: \\
				Given operations $W1\ R2$, \textbf{transaction $T2$ reads the effect of transaction $T1$}
				&&&&& If the operations are read then write: \\
				Given operations $R1\ W2$, \textbf{transaction $T1$ does not read the effect of transaction $T2$}
				&&&&& If the operations are both write: \\
				Given operations $W1\ W2$, \textbf{transaction $T2$ overwrites transaction $T1$}
	
		&&& If and only if all conflict actions in a schedule induce the same priority of transactions, then the schedule is serializable
		
		&&& Precedence graph is created by analyzing each conflict action and adding directional edges to transactions in their induced orders
			&&&& An acyclic precedence graph can be navigated in any topological order to create a serial schedule
			
		&&& \textbf{Conflict-serializable schedule:} Consistent serializable schedule which has an acyclic precedence graph
			&&&& Subset of serializable schedules

\clearpage
\end{easylist}
\subsection{Isolation}
	\label{subsec:isolation}
\begin{easylist}

& \textbf{Isolation:} Property of a transaction where, if it is aborted, other transactions should not be affected
	&& \textbf{Dependent:} Transaction which reads from or writes to a data item, with the result dependent on the write action of another transaction
		&&& Conflict actions which create a dependency: $WW$ or $WR$
		&&& E.g. If the order of actions is $W1\ R2$, then $T2$ depends on $T1$

& \textbf{Cascading abort:} Abort action which causes a different transaction to also abort
	&& Invalidates the isolation property
	&& E.g. Given concurrent transactions $T1$ and $T2$:
		&&& $W1\ R2\ ABORT(1)$ means that the value before $W1$ must be restored, which invalidates $R2$. Therefore, $T2$ must be aborted in addition to $T1$.
		&&& $W1\ W2\ ABORT(1)$ means that the value before $W1$ must be restored, which loses the write by $W2$. Therefore, $T2$ must be aborted in addition to $T1$.

& \textbf{Recoverable schedule:} Isolated schedule where a transaction can only commit after the transactions it depends on have committed
	&& Cascading aborts are possible, but are safe to do

    && \textbf{Avoid Cascading Aborts (ACA) schedule:} Isolated schedule which avoids cascading aborts by waiting to access a resource with an uncommitted write, until the action is committed
        &&& Subset of recoverable schedules
        &&& All dependencies are replaced with waits
        &&& The only conflict action allowed is $RW$
        	&&&& If a schedule has $WW$ or $WR$ conflict actions, then it violates ACA

\clearpage
\end{easylist}
\subsection{Enforcing Consistency and Isolation}
	\label{subsec:enforcing-consistency-and-isolation}
\begin{easylist}

& Resource locks:
	&& \textbf{Share lock (S-lock):} Lock on a resource during a read operation
		&&& Notation: $S(T)$
	&& \textbf{Exclusive lock (X-lock):} Lock on a resource during a write operation
		&&& Notation: $X(T)$
		&&& \textbf{Upgrade:} Action where a lock is changed from share to exclusive, due to a transaction requesting an exclusive lock while holding a share lock on the same resource
	&& Unlock notation: $U(T)$
	&& Locking by itself does not guarantee serializability
	&& \textbf{Lock table:} Hash table of lock entries
		&&& \textbf{Lock entry:} Data structure representing a single resource which maintains a list of transactions holding locks and a wait queue
		&&& \textbf{On lock release (OID/XID):} Operation where the next available locks in the wait queue are processed

& \textbf{2PL:} Resource access lock protocol which ensures conflict serializability (consistency)
	&& Rules: For each transaction:
		&&& Before reading, a share lock must be activated
		&&& Before writing, an exclusive lock must be activated
		&&& All unlocks must happen after all lock operations
			&&&& \textbf{Lock point:} Time in a schedule at which all required locks have been acquired by a transaction
	&& Does not enforce isolation

& \textbf{Strict 2PL:} Resource access lock protocol which ensures conflict serializability (consistency) and ACA (isolation)
	&& Rules: For each transaction:
		&&& Before reading, a share lock must be activated
		&&& Before writing, an exclusive lock must be activated
		&&& Unlocks only happen when committing
	&& E.g. Schedule which violates string 2PL: $R2(A)\ W1(A)\ W2(A)$

\clearpage
\end{easylist}
\subsection{Deadlocks}
	\label{subsec:deadlocks}
\begin{easylist}

& \textbf{Deadlock:} Situation where two transactions hold locks while waiting for the other transaction to release their lock
	&& Can occur in 2PL or strict 2PL

\end{easylist}
\subsubsection{Detection Strategies}
	\label{subsubsec:detection-strategies}
\begin{easylist}

& \textbf{Waits-for-graph:} Deadlock detection strategy where a directed graph is maintained between transactions waiting for another transaction to release a lock, and if a cycle (deadlock) is detected, a transaction in the cycle is aborted
	&& Aborted transaction is chosen by criteria such as fewest locks, least work done, farthest from completion, etc.

& \textbf{Timeout:} Deadlock detection strategy where a transaction which exceeds a certain duration is aborted

\end{easylist}
\subsubsection{Prevention Strategies}
	\label{subsubsec:prevention-strategies}
\begin{easylist}

& Each transaction is assigned a unique timestamp; waits can only occur in one temporal direction
	&& \textbf{Wait-die:} Deadlock prevention strategy where older transactions may wait for newer transactions
	&& \textbf{Wound-wait:} Deadlock prevention strategy where newer transactions may wait for older transactions
	&& If a transaction violates this, the newer transaction is aborted and restarted after some time with the original timestamp

\end{easylist}
\subsection{B+ Tree Concurrency}
	\label{subsec:b-tree-concurrency}
\begin{easylist}

& B+ tree indices also require locking

& Locking a node also locks the subtree

& \textbf{Crabbing locking:} B+ tree locking strategy which releases ancestor locks as child locks are obtained, `crabbing' down the tree
	&& Search: Release a lock on a node when a child node of it is locked
	&& Insert/delete: Release a lock on any ancestor nodes when a child node of it is locked, and the child is not full
		&&& Reason: Any split in the index will propagate to the non-full child, then stop

\end{easylist}
\clearpage
