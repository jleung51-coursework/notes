%
% BUS 361: Project Management - A Course Overview
% Section: Human Resources
%
% Author: Jeffrey Leung
%

\section{Human Resources}
	\label{sec:human-resources}
\begin{easylist}

& Planning resourcing:
	&& Create positions with skills, requirements
	&& Chart hierarchy
	&& Procure and assign resources

& \textbf{Project team:} Group of two or more people who share goals, are interdependent, have complementary skills, and are mutually accountable

& Characteristics of effective teams:
	&& Commitment to a goal or purpose
	&& Morale, team spirit
	&& Synergistic work
	&& Complementary skills
	&& Support

& \textbf{Tuckman's Team Development Stages:}
	&& \textbf{Formation:} Stage of team development when the team gets to know each other with awkwardness and miscommunication
		&& Agreed-upon points: Structure, goals, direction, roles
	&& \textbf{Storming:} Stage of team development when the team disagrees and resolves conflicts about the abilities to meet the goal
	&& \textbf{Norming:} Stage of team development when the team communicates well, resolves problems, becomes comfortable, and accepts teamwork
	&& \textbf{Performing:} Stage of team development when the team works independently and adaptively, can solve problems well, and has high morale
	&& \textbf{Adjourning:} Stage of team development when the team is recognized for achievements, says personal goodbyes

& Team development techniques:
	&& Team building activities
	&& Training
	&& Delegation
	&& Reward and recognition systems

\end{easylist}
\subsection{Motivation}
	\label{subsec:motivation}
\begin{easylist}

& \textbf{Motivation:} Intensity, direction, and persistence towards a goal

& \textbf{Extrinsic motivation:} Motivation based on earning a reward or avoiding a punishment
& \textbf{Intrinsic motivation:} Motivation based on a personal and internal reward

& \textbf{Maslow's Hierarchy of Needs:} Priorization of needs which must be fulfilled in the order of physiological, security, social, esteem, and self-actualization
& \textbf{McLelland's Three-Needs Theories:} Motivations are derived from aspirations toward achievement, power, or affiliation, one of which is primary
& \textbf{Theory X, Y:} $X$: People dislike work and responsibility and must be coerced, $Y$: People enjoy work, are creative, and want autonomy and responsibility
& \textbf{Expectancy Theory:} Motivation of effort results in increased performance which leads to higher value rewards/results
& \textbf{Adams' Equity Theory:} Motivation comes from perceived fairness, and inequities in input or output ratios will affect motivation
	&& Social comparison processes skews perceptions of equity

& Conflict:
	&& \textbf{Task conflict:} Conflict over project goals
	&& \textbf{Process conflict:} Conflict over the process of how work is carried out
	&& \textbf{Relationship conflict:} Conflict over interpersonal relationships
	&& \textbf{Functional conflict:} Conflict which improves the team (e.g. low level of task/process conflict)
	&& \textbf{Dysfunctional conflict:} Conflict which is harmful to the team (e.g. relationship conflict, high level of task/process conflict)

\end{easylist}
\clearpage
