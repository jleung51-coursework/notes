%
% CMPT 320: Social Implications of a Computerized Society - A Course Overview
% Section: Intellectual Property
%
% Author: Jeffrey Leung
%

\section{Intellectual Property}
	\label{sec:intellectual-property}
\begin{easylist}

& \textbf{Intellectual property:} Right to owning the underlying concept of a creation
	&& Positive right (see \hyperref[sec:ethics]{Positive/claim right})
	&& Contrasted with the owner of a product having a negative right and property right over their physical copy

& \textbf{Copyright:} Automatic exclusive legal right to (re)produce, distribute, create derivatives of, and perform work (i.e. the expression of the idea)
	&& Protects creative works and creative expression (but not the physical copy)
	&& Promotes innovation by allowing creator to benefits
	&& Expires over time and work enters public domain
	&& E.g. Books, articles, movies, songs, art
	&& Does not protect facts, concepts, processes, algorithms, ideas
	&& Enforcement tactics:
		&&& Software: Activation numbers, expiration dates, DRM
		&&& Suing ISPs with subscribers who violate copyright
		&&& Suing new technology to suppress or delay it (e.g. CD burners)
		&&& Taxing products which can make copies
	&& EU approved copyright law requiring media platforms to detect copyright material using filters during upload
	&& Modern software alternatives to copyright:
		&&& Open source, freeware, shareware
		&&& Licences (e.g. Creative Commons)
		&&& \textbf{Copyleft:} Right to use, modify, and distribute software as long as the same permissions are given to their products

& \textbf{Patent:} Approved right to make, use, and sell an invention
	&& Laws of nature/mathematics cannot be patented
	&& Software can be patented (e.g. encryption, compression, copy protection)
	&& E.g. A CPU with a specific architecture, a phone with a specific screen design, one-click shopping, personalized product recommendations, privacy controls
	&& Cons: New innovations are only usable by one group; new products have difficulty avoiding infringement

& \textbf{Fair use:} Limited ability to use copyrighted material without permission
	&& 4 key factors:
		&&& Purpose (must be educational, not for profit)
		&&& Nature of work (fiction vs. factual report)
		&&& Amount and substantiality
		&&& Effect on value or potential market
	&& E.g. Criticism, parodies, news, teaching, research
	&& Arguments against international patent enforcement to support global development

\end{easylist}
\clearpage