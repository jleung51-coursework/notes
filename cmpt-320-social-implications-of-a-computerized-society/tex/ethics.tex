%
% CMPT 320: Social Implications of a Computerized Society - A Course Overview
% Section: Ethics
%
% Author: Jeffrey Leung
%

\section{Ethics}
	\label{sec:ethics}
\begin{easylist}

& \textbf{Fundamental ethical principles/frameworks:} Ethical systems by which to judge the morality of an act

& \textbf{Universalizability:} Judging an act by the theoretical consequences of everyone performing that act
	&& \textbf{Immanuel Kant's categorical imperative:} Act according to the maxim that you would wish all other rational people to follow, as if it were a universal law

& \textbf{Consequentialism:} Judging an act purely by the resulting consequences, rather than a sense of intrinsic duty or the character of the behaviour
	&& \textbf{Utilitarianism:} Judging an act by maximizing the consequences, defined in terms of the sum of the individual utility to specific parties
		&&& Not based on selfish motivations; calculated with greater good
		&&& To evaluate, use an arbitrary quantitative analysis of utility for each party (e.g. see table~\ref{tab:util-hitler})

\end{easylist}
\begin{table}[!htb]
	\centering
	\caption{Utilitarian Analysis of Killing Hitler}
	\label{tab:util-hitler}
	\begin{tabular}{ l | r r r | r }
		& \multicolumn{3}{ | c | }{Entity} \\
		Options & You & Hitler & Rest of the World & Total \\
		\hline
		Shoot & 100 & -1000 & 100,000 & 99,100 \\
		Don't shoot & -100 & 0 & -100,000 & -100,100 \\
	\end{tabular}
\end{table}
\begin{easylist}

& \textbf{Deontology:} Judging an act purely by a sense of intrinsic duty and the character of the behaviour, rather than the resulting consequences
	&& Associated with basic rights
	&& E.g. The Ten Commandments are intrinsic behavioural rules
	&& Immanuel Kant's deontic morality: Using the categorical imperative and our reasoning to determine morality and to treat people as ends, not means

& \textbf{Fairness:} Judging an act by the equality of how people are treated
	&& \textbf{Equality:} Distribution which is quantitatively the same for each entity
	&& \textbf{Need:} Distribution based on what each entity requires or is lacking
	&& \textbf{Merit/equity:} Distribution based on the effort and contribution by each entity

& \textbf{Social contract:} Agreement for the rules of a group of people, made between rational people who are free not to join
	&& \textbf{Veil of ignorance:} Concept of decision-making while imagining they do not know anything about their own characteristics in a social order
		&&& Preserves impartiality

& \textbf{Rights-based:} Basic system of rights and freedoms relating to yourself or others
	&& \textbf{Negative right:} Entitlement to act freely without interference (e.g. freedom of speech or assembly)
	&& \textbf{Positive/claim right:} Entitlement to a service or benefit (e.g. health care, education)

& \textbf{Canadian Charter of Rights and Freedoms} protects the freedoms of:
	&& Conscience and religion
	&& Thought, belief, opinion and expression, including freedom of the press and other media of communication
	&& Peaceful assembly
	&& Association
& \textbf{UN Declaration of Child Rights} upholds the principles that a child shall:
	&& Be given opportunities to develop physically, mentally, morally, spiritually and socially in conditions of freedom and dignity
	&& Be entitled from his birth to a name and nationality
	&& Enjoy the benefits of social security
	&& Be given care including prenatal and postnatal care
	&& Be provided with nutrition, housing, recreation and medical services

\end{easylist}
\clearpage