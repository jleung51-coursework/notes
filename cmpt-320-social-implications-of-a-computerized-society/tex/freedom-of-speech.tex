%
% CMPT 320: Social Implications of a Computerized Society - A Course Overview
% Section: Freedom of Speech
%
% Author: Jeffrey Leung
%

\section{Freedom of Speech}
	\label{sec:freedom-of-speech}
\begin{easylist}

& \textbf{Freedom of speech:} Ability to express opinions without restraint
	&& Distinguished from action (e.g. inciting illegal acts, libel, or threats are not allowed)
	&& \textbf{Chilling:} Inhibition of a natural or legal right
		&&& Laws cannot chill charter rights such as freedom of speech
	&& Allows discussing and debating to improve social welfare
	&& Be minimally restrictive (e.g. adults should not be restricted to child-appropriate content)

& Software protection:
	&& In 1990s, US government restricted publishing encryption software; judge decided that the source code was classified as speech

& \textbf{Publisher/broadcaster:} Service which creates and manages content
	&& Liable for content
	&& Holds the right to freedom of expression and right not to publish
& \textbf{Common carrier:} Service which hosts content or provides benefits from other entities
	&& Not liable for content
	&& Is required to give universal access
& Generally, providers are not liable for illegal uses as long as the service has substantive legal use

& Improvement in technology and stronger government presence allows for greater censorship

& \textbf{Spam:} Unwanted targeted electronic communication
	&& Free speech is not an issue
	&& Possible economic method for deterrence: Charging a sender to pay
		&&& Deters lower-income groups
	&& Technological method for deterrence: Filters
	&& Can-Spam bill by US Congress: Requires opt-out option, cannot disguise \textit{From} field

\end{easylist}
\clearpage