%
% STAT 100: Chance and Data Analysis - A Course Overview
% Section: Data Collection
%
% Author: Jeffrey Leung
%

\section{Data Collection}
	\label{sec:data-collection}
\subsection{Methods}
	\label{subsec:data-collection:methods}
\begin{easylist}

	& \emph{Census:} Survey which collects information from all individuals of a population
		&& Process important?
		&& Not always viable because it may be:
			&&& Expensive
			&&& Time-consuming
			&&& Impossible (e.g. taking a census of everyone in the world)
		&& E.g. Taking a census of all Canadian citizens
		
	& \emph{Sample survey:} Survey which collects information from a select group of individuals smaller than and representative of the population
		&& Relatively greater data quality due to smaller samples
		&& E.g. Taking a sample survey of which political party for which people plan to vote
		
	& \emph{Observational study:} Data collection method in which data is collected through observation without interference
		&& Concludes an correlation or lack thereof between two variables
			&&& Lurking variable(s) (see ref) may connect the two variables and explain the association %TODO reference
		%TODO
		&& E.g. Students were given the option to attend an optional tutorial session. Those who attended the tutorial session were more likely to receive a higher mark on their exam. %TODO
	
	& \emph{(Randomized) Experiment:} Data collection method in which individuals are chosen randomly to... %TODO
		&& Not all factors can be studied through randomized experiments
		&& E.g. Some students are chosen at random to be given extra tutorial sessions. They receive...
		%TODO
		
		%TODO references to the subsections below

\end{easylist}
\subsection{Sampling}
	\label{subsec:data-collection:sampling}
\begin{easylist}

	& \emph{Random sample:} %TODO
	
	& \emph{Simple random sample:} Set of individuals selected from the population such that each individual has an equal chance of being selected
		&& Process: To select $y$ individuals from a group of $x$, number each individual
		&& %TODO
		
\end{easylist}
\subsection{Sampling Errors}
	\label{subsec:data-collection:sampling-errors}
\begin{easylist}

	& Non-sample errors:
	
		&& \emph{Response error:} Inaccurate/untruthful response which skews data
			&&& E.g. An experiment which asks a question about a morally shameful activity may have a response error due a tendency to answer with the morally best choice, so as to give a better impression
		&& \emph{Non-response error:} Lack of response which skews data
			&&& E.g. An experiment which offers an optional response to a survey, such as through email, may have a non-response error due to some people not responding because of not checking email, laziness, apathy
			&&& No standard way to determine a good response rate
				&&&& Theoretically, 75\% and above is acceptable
			&&& Analyze the response rate if possible, and the reason for a low response rate
		&& \emph{Question wording:} Syntax/keywords/details of a question in a survey which skews data
		
	& Voluntary response error:
		&& To explain, calculate the (low) response rate and offer a possible explanation

\end{easylist}
\subsection{Design of an Experiment}
	\label{subsec:data-collection:design-of-an-experiment}
\begin{easylist}

	& \emph{Explanatory variable:} Factor which is hypothesized to affect another factor
		&& \emph{Treatment:} Unique combination of one of each type of explanatory variable
			&&& $Number\ of\ treatments = num\ of\ variable_{1} \times num\ of\ variable_{2} \times \cdots$
			&&& Table of all possible treatments should be written
			&&& Sample is assigned equally and randomly into the number of treatments
	& \emph{Response variable:} Resultant factor which is hypothesized to be affected by another factor
	
	& Experiment diagram + block diagram %TODO
	
	& E.g. An experiment was conducted to find out if the length and/or repetitions of a TV commercial affect the desire to buy a product. 20 subjects were chosen for the experiment. \smallskip \\
	
	\emph{Explanatory variables:} \\
	\indent Length of the TV commercial (1 minute / 5 minutes / 10 minutes) \\
	\indent Number of repetitions of the TV commercial (1 time / 3 times / 5 times) \\
	\emph{Response variable:} \\
	Desire to buy the product in the TV commercial (Scale from 1-5 where 1 = Do not want to buy) \\
	\emph{Treatments:} \\
	$3 \times 3 = 9$ treatments (see table~\ref{tab:experiment-length-repetitions-commercial-treatments}) \\
	
	
	
	\Deactivate
	\begin{table}[!htb]
		\centering
		\caption{Treatments of a Experiment on Length and Repetitions of a Commercial}
		\label{tab:experiment-length-repetitions-commercial-treatments}
		\begin{tabular}{ r r }
			Length (mins): & Repetitions: \\
			\hline
			 1 & 1 \\
			   & 3 \\
			   & 5 \\
			 5 & 1 \\
			   & 3 \\
			   & 5 \\
			10 & 1 \\
			   & 3 \\
			   & 5
		\end{tabular}
	\end{table}
	\Activate
		
\end{easylist}
\clearpage