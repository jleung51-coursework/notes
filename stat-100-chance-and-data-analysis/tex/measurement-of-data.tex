%
% STAT 100: Chance and Data Analysis - A Course Overview
% Section: Measurement of Data
%
% Author: Jeffrey Leung
%

\section{Measurement of Data}
	\label{sec:measurement-of-data}
\subsection{Introduction}
	\label{subsec:measurement-of-data:introduction}
\begin{easylist}

	& \emph{Measurement:} Collection of quantitative data
	
	& \emph{Unit of measurement:} System of set values to quantify data
		&& Comparisons between measurements must use the same units
	
	& \emph{Instrument:} Tool to measure a quantitative characteristic of an individual
		&& E.g. Ruler, scale
		
\end{easylist}
\subsection{Imprecision of Measurements}
	\label{subsec:measurement-of-data:imprecision-of-measurements}
\begin{easylist}

	& \emph{Uncertainty (measurement):} Possible error in a measurement due to imprecision of an instrument
	& \emph{Bias (measurement):} Difference between the measured value and the true value of a quantitative characteristic
	& Uncertainty and bias can be reduced by using instruments with higher precision

	& \emph{Random error:} Variations between repeated measurements on the same individual
		&& E.g. Surveying 100 people out of a population of 1000; recording the weight of a live, energetic pig
		&& Can be reduced by averaging multiple repeated measurements
		&& The less the random error, the more reliable the measurement
		
	& \emph{Variance:} Unreliability of a measurement (calculated from multiple measurements
		&& Calculation:
		\begin{displaymath}
			Variance =
			\frac
			{
				\sum\limits_{i=0}^{n}
				(x_{i} - \bar{x})^{2}
			}
			{
				n - 1
			}
		\end{displaymath}
		\Deactivate
		\begin{center}
			\begin{tabular}{ l r @{ = } l }
				where & $n$ & number of measurements \\
				& $x_{i}$ & measurement number $i$ \\
				& $\bar{x}$ & average of measurements
			\end{tabular}
		\end{center}
		\Activate
		
		&& E.g. Find the variance of the measurements 220 lbs, 224 lbs, 217 lbs, and 227 lbs.
		
		\Deactivate
		\begin{center}
			\begin{tabular}{ r @{ = } l }
				$n$ & 4 \\
				$\bar{x}$ & $\frac{220 + 224 + 217 + 227}{4} = \frac{888}{4} = 222$
			\end{tabular}
		\end{center}
		\Activate
		
		\Deactivate
		\begin{IEEEeqnarray}{ r C l }
			Variance
			& = & \frac
			{
				\sum\limits_{i=0}^{n}
				(x_{i} - \bar{x})^{2}
			}
			{
				n - 1
			} \\
			& = & \frac
			{
				(220-222)^{2} + (224-222)^{2} + (217-222)^{2} + (227-222)^{2}
			}
			{
				4-1
			} \\
			& = & \frac
			{
				4 + 4 + 25 + 25
			}
			{
				3
			} \\
			& = & \frac{58}{3}
		\end{IEEEeqnarray}
		\Activate

\end{easylist}
\subsection{Relevancy of a Measurement}
	\label{subsec:measurement-of-data:relevancy-of-a-measurement}
\begin{easylist}

	& A valid measurement should be a relevant representation of the property to be studied, and not some other property

	& Rates are a standard measure of comparison, while counts are not
		&& E.g. Out of 50 people, the morning class had 20 attendees. Out of 100 people, the evening class had 30 attendees. \smallskip \\
		Comparing the number/count of subjects, the evening class (30) had greater attendance than the morning class (20). \\
		Comparing the rates of subjects, the morning class ($\frac{20}{50} = 0.4$) had greater attendance than the evening class ($\frac{30}{100} = 0.3$).

\end{easylist}
\clearpage