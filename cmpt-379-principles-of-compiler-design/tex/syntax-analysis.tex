%
% CMPT 379: Principles of Compiler Design - A Course Overview
% Section: Syntax Analysis (Parsing)
%
% Author: Jeffrey Leung
%

\section{Syntax Analysis (Parsing)}
	\label{sec:syntax-analysis}
\begin{easylist}

& \textbf{Syntax analysis (parsing):} Translation of a set of strings into a valid structure using a grammar
	&& \textbf{Language:} Description of a valid string of tokens

& Symbols and rules:
	&& \textbf{Terminal symbol:} Token or base symbol which does not represent any other symbols
	&& \textbf{Non-terminal symbol:} Set of strings potentially consisting of one or more terminal or non-terminal symbols
	&& \textbf{Production/Rule/Production rule:} Function on a non-terminal symbol which generates a sequence of terminal and/or non-terminal symbols
		&&& \textbf{Expansion:} Sequence of symbols resulting from the application of a production rule

\end{easylist}
\subsection{Context-Free Grammars}
	\label{subsec:syntax-analysis:context-free-grammars}
\begin{easylist}

& \textbf{Context-Free Grammar (CFG):} Recursive notation for the structure of a valid syntax; subset of regular grammars
	&& Generates a \textit{Context-Free Language (CFL)}
	&& Defined by a set of non-terminal states $N$, a start state $S \in N$, a set of terminal states $T$, and a set of productions to create a set of states
		&&& Each state is a string consisting of non-terminal and/or terminal symbols
		&&& If multiple productions exist from the start state, then in order to backtrack from a terminal state to the start state, create a new start state $S'$ which has a single production to the original state state
		&&& Productions can be implemented using a transition matrix
	&& E.g. Given grammar $\bigg\{ (^i)^i | i \geq 0 \bigg\}$ (i.e. The number of opening brackets must match the number of closing brackets) the productions are:
		&&& $S \rightarrow \epsilon$
		&&& $S \rightarrow (S)$
		&&& $N = \{ S \}$
		&&& $T = \big\{ (, ) \big\}$
	&& E.g. Given language $\bigg\{ wcw^R | w \in (a|b)* \bigg\}$, a CFG which produces this language is $S \rightarrow aSa | bSb | c$
	&& E.g. Given language $\bigg\{ a^n b^m c^m d^n | n \geq 1, m \geq 1 \bigg\}$, a CFG which produces this language is $S \rightarrow aSd | aAd; A \rightarrow bAc | bc$
	&& \textbf{Ambiguous:} CFG which can produce multiple unique parse trees
		&&& Unacceptable for programming languages
		&&& Prevention techniques:
			&&&& Rewriting grammar unambiguously
			&&&& Separating non-terminal states by different precedence levels
				&&&&& E.g. $E \rightarrow E - E; E \rightarrow T; T \rightarrow T / T$
				&&&&& E.g. $A \rightarrow A - B; A \rightarrow B$
			&&&& \textbf{Left/Right associative:} Property of an operator specifying which of multiple operations with the same precedence should be evaluated first
	&& \textbf{Recursive in non-terminal $X$:} CFG where, in one or more productions, $X$ can derive a sequence of symbols including $X$
		&&& \textbf{Left recursive in non-terminal $X$:} CFG where, in one or more productions, $X$ can derive a sequence of symbols which begins with $X$
		&&& \textbf{Right recursive in non-terminal $X$:} CFG where, in one or more productions, $X$ can derive a sequence of symbols which ends with $X$

\end{easylist}
\subsection{Parsing}
	\label{subsec:syntax-analysis:parsing}
\begin{easylist}

& \textbf{Parse tree:} Tree-shaped sequence of CFG productions used to generate a valid derivation of symbols, with the root being the start symbol and children being productions of the parent (i.e. for a given production $X \rightarrow Y_1 \dotsc Y_n$, the children of $X$ are $Y_1 \dotsc Y_n$)
	&& \textbf{Leftmost derivation:} Sequence of CFG productions to create a parse tree by continually processing the leftmost non-terminal state
	&& \textbf{Rightmost derivation:} Sequence of CFG productions to create a parse tree by continually processing the rightmost non-terminal state
	&& E.g. For the production rules in table~\ref{tab:parse-tree-example-1-prod-rules}, the parse tree of $id + id * id$ is shown in figure~\ref{fig:parse-tree-example-1}
	&& E.g. For the production rules in table~\ref{tab:parse-tree-example-2-prod-rules}, the parse tree of $id + id * id$ is shown in figure~\ref{fig:parse-tree-example-2}

\end{easylist}
\begin{figure}[!htb]
	\caption{Parse Tree Example 1: Production Rules}
	\label{tab:parse-tree-example-1-prod-rules}
	\begin{center}
		\begin{tabular}{ l }
			$E \rightarrow E + E$ \\
			$E \rightarrow E * E$ \\
			$E \rightarrow id$
		\end{tabular}
	\end{center}
\end{figure}
\begin{easylist}

\end{easylist}
\begin{figure}[!htb]
	\caption{Parse Tree Example 1}
	\label{fig:parse-tree-example-1}
	\begin{center}
		String: $id + id * id$ \\[1em]
		\begin{forest}
			[$E$
				[$E$
					[$E$[$id$]]
					[$+$]
					[$E$[$id$]]
				]
				[$*$]
				[$E$
					[$id$]
				]
			]
		\end{forest}
	\end{center}
\end{figure}
\begin{easylist}

\end{easylist}
\begin{figure}[!htb]
	\caption{Parse Tree Example 2: Production Rules}
	\label{tab:parse-tree-example-2-prod-rules}
	\begin{center}
		\begin{tabular}{ l }
			$E \rightarrow T + E$ \\
			$E \rightarrow T$ \\
			$T \rightarrow int$ \\
			$T \rightarrow int * T$ \\
			$T \rightarrow (E)$
		\end{tabular}
	\end{center}
\end{figure}
\begin{easylist}

\end{easylist}
\begin{figure}[!htb]
	\caption{Parse Tree Example 2}
	\label{fig:parse-tree-example-2}
	\begin{center}
		String: $int * int + int$ \\[1em]
		\begin{forest}
			[$E$
				[$T$
					[$int$]
					[$*$]
					[$T$[$int$]]
				]
				[$+$]
				[$E$[$T$[$int$]]]
			]
		\end{forest}
	\end{center}
\end{figure}
\begin{easylist}

& \textbf{Pushdown automaton (PDA):} \hyperref[subsec:finite-state-automata]{Finite State Automata} system which utilizes a stack to parse a CFL
	&& Notations:
		&&& Finite set of states: $S$
		&&& Start state: Outlined
		&&& Final/accepting state: Double-outlined
		&&& Production rules: $P$
	&& Uses a stack structure to hold symbols
	&& For each CFG, there exists an equivalent PDA

& \textbf{Shift-reduce parsing:} Method of bottom-up parsing which reduces a string to a start symbol, using reversed production rules
	&& Uses a left substring containing terminal or non-terminal symbols, and a right substring containing as-of-yet unprocessed symbols
	&& Starts with the entire string in the right substring
	&& Right substring utilizes a stack with the following options:
		&&& \textbf{Shift:} Retrieving a symbol from the stack (right substring) and placing it on the left substring
		&&& \textbf{Reduce:} Removing one or more symbols from the stack (right substring), processing them with a reversed production rule, and returning the result to the stack
	&& Example:
	\end{easylist}
	\begin{align*}
		\textrm{Stack} & | \textrm{Input} & \textrm{Action} \\
		\hline
		& | x_1 x_2 x_3 x_4 & \textrm{Begin}\\
		x_1 & | x_2 x_3 x_4 & \textrm{Shift } x_1 \\
		x_1 & | y_1 x_4 & \textrm{Reduce } x_2 x_3 \textrm{ to } y_1 \\
		x_1 y_1 & | x_4 & \textrm{Shift } y_1 1\\
		x_1 y_1 & | y_2 & \textrm{Reduce } x_4 \textrm{ to } y_2 \\
		x_1 y_1 y_2 & | & \textrm{Shift } y_2
	\end{align*}
	\begin{easylist}

	&& \textbf{Shift-reduce conflict:} Situation in shift-reduce parsing where either a shift or reduce can create valid parses
		&&& Solution: Precedence/associativity declarations
	&& \textbf{Reduce-reduce conflict:} Situation in shift-reduce parsing where multiple different reduces can create valid parses
		&&& Solution: Rewrite grammar to no longer be ambiguous
	&& \textbf{Handle:} Sequence of symbol(s) at the top of the stack which is a valid reduction allowing further reductions towards the start symbol
		&&& \textbf{Prune:} Action of reducing a handle and pushing the result onto the stack
		&&& To find a handle, heuristics are used; no efficient algorithms exist

\end{easylist}
\clearpage
