\documentclass[10pt, oneside, letterpaper, titlepage]{article}

\usepackage{amsmath}
\usepackage[ampersand]{easylist}
	\ListProperties(
		Progressive*=5ex,
		Space=5pt,
		Space*=5pt,
		Style1*=\textbullet\ \ ,
		Style2*=\begin{normalfont}\begin{bfseries}\textendash\end{bfseries}\end{normalfont} \ \ ,
		Style3*=\textasteriskcentered\ \ ,
		Style4*=\begin{normalfont}\begin{bfseries}\textperiodcentered\end{bfseries}\end{normalfont}\ \ ,
		Style5*=\textbullet\ \ ,
		Style6*=\begin{normalfont}\begin{bfseries}\textendash\end{bfseries}\end{normalfont}\ \ ,
		Style7*=\textasteriskcentered\ \ ,
		Style8*=\begin{normalfont}\begin{bfseries}\textperiodcentered\end{bfseries}\end{normalfont}\ \ ,
		Hide1=1,
		Hide2=2,
		Hide3=3,
		Hide4=4,
		Hide5=5,
		Hide6=6,
		Hide7=7,
		Hide8=8 )
\usepackage[T1]{fontenc}
\usepackage{forest}
\usepackage{geometry}
	\geometry{margin=1.2in}
\usepackage{graphicx}
	\graphicspath{ {img/} }
\usepackage[colorlinks=true, linkcolor=blue]{hyperref}
\usepackage{listings}
\usepackage{lmodern} % Allows the use of symbols in font size 10; http://ctan.org/pkg/lm
\usepackage{multirow}
\usepackage{textcomp} % Allows the use of \textbullet with the font
\usepackage{verbatim}

\renewcommand{\arraystretch}{1.2}
\renewcommand{\familydefault}{\sfdefault}

\begin{document}

\section{LL(1) Parsing}
\subsection{Introduction}
\begin{easylist}

& \textbf{LL(1) parsing:} Top-down parsing algorithm which creates a left-to-right leftmost derivation requiring only 1 symbol of lookahead
	&& Requires a grammar which is:
		&&& Unambiguous
		&&& Left-factored
		&&& Free of left-recursion
	&& Only one possible production can exist given the next token

& Grammar validity:
	&& A grammar is LL(1) if and only if, for each pairing of a non-terminal symbol with a terminal symbol, only one possible production rule can be applied
	&& A grammar is not LL(1) if and only if there exists at least one pairing of a non-terminal symbol and a terminal symbol which can expand to multiple production rules (e.g. $A \rightarrow bc; A \rightarrow bd$)
	
\end{easylist}
\subsection{Predictive Parsing Tables}
\begin{easylist}

& \textbf{Predictive parsing table:} Table which correlates the leftmost non-terminal symbol and the next terminal symbol with the only possible production
	&& No cell can have more than one string as the expansion to the rule
	&& Empty cells mean parsing errors
	&& Uses lookahead to resolve conflicts between multiple possible production rule applications for the same symbol
	&& Layout: See figure~\ref{tab:predictive-parsing-table-layout}

\end{easylist}
\begin{figure}[!htb]
	\caption{Predictive Parsing Table Layout}
	\label{tab:predictive-parsing-table-layout}
	\begin{center}
		\begin{tabular}{ r r | l l }
			& & \textbf{Terminal symbols} & \$ \\
			\hline
			\textbf{Non-terminal Symbols} & \textit{Symbol} & \textit{Production result}
		\end{tabular}
	\end{center}
\end{figure}
\begin{easylist}

& To create a predictive parsing table:
	&& For each non-terminal symbol $X$ on the left column:
		&&& If there is a production rule where $X$ expands to epsilon ($\epsilon$), then:
			&&&& For each terminal symbol which is in the $FOLLOW$ of symbol $X$, write $\epsilon$
		&&& If there is a production rule where $X$ expands to a string which is not epsilon ($\epsilon$), then:
			&&&& For each terminal symbol which is in the $FIRST$ of symbol $X$, write the expansion of the valid production rule

& E.g. In figure~\ref{tab:ll1-conflict-resolution}, $Y \rightarrow \epsilon$ is the valid production rule to be chosen when $+, ),$ or $\$$ are the lookahead symbol

\end{easylist}
\begin{figure}[!htb]
	\caption{LL(1) Conflict Resolution with $FOLLOW$}
	\label{tab:ll1-conflict-resolution}
	\begin{center}
		\begin{tabular}{ r l }
			Production Rules
			& $E \rightarrow TX$ \\
			& $X \rightarrow \epsilon$ \\
			& $X \rightarrow +E$ \\
			& $T \rightarrow (E)$ \\
			& $T \rightarrow id Y$ \\
			& $Y \rightarrow *T$ \\
			& $Y \rightarrow \epsilon $ \\
			\hline
			$FOLLOW(Y)$
			& $= FOLLOW(T)$ \\
			& $= ( FIRST(X) - \{\epsilon\} ) + FOLLOW(E)$ \\
			& $= \{ +, ), \$ \}$
		\end{tabular}
	\end{center}
\end{figure}
\begin{easylist}

\clearpage

\end{easylist}
\subsection{LL(1) Parsing Trace}
\begin{easylist}

& LL(1) parsing trace:
	&& Stack is used to keep track of non-terminal symbols
	&& Stack and input strings are terminated with the end of input symbol (\$)
	&& Layout: See figure~\ref{tab:ll1-parsing-trace-example}

\end{easylist}
\begin{figure}[!htb]
	\caption{LL(1) Parsing Trace Example}
	\label{tab:ll1-parsing-trace-example}
	\begin{center}
		\begin{tabular}{ l | l | l }
			\textbf{Stack} & \textbf{Input} & \textbf{Action} \\
			\hline
			\textit{string \$} & \textit{string \$} & \textit{Production rule to expand OR} \\
			&& \textit{$\epsilon$ OR} \\
			&& \textit{Terminal $\alpha$ OR} \\
			&& \textit{acc}
		\end{tabular}
	\end{center}
\end{figure}
\begin{easylist}

& To create an LL(1) parsing trace:
	&& For the leftmost input symbol and the leftmost stack symbol:
		&&& If the leftmost stack symbol is a non-terminal symbol, then use the leftmost input and stack symbols to find the corresponding production rule expansion from the predictive parsing table, and:
			&&&& If the expansion is not an epsilon, then replace the leftmost stack symbol with the expansion
			&&&& If the expansion is an epsilon ($\epsilon$), then remove the leftmost stack symbol
		&&& If the leftmost stack symbol is a terminal symbol and the leftmost stack and input symbols match, then consume both of them with the action \textit{Terminal}
		&&& If the leftmost stack and input symbols are the end of input (\$), then accept the parse

\end{easylist}

\end{document}
